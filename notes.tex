Your article reads more like a note with same potentially interesting derivations than a full-fledged article.  First, the model is static and simplistic.  You make a number of strong assumptions regarding market structure that may or may not hold. For example, the deposit market is a monopsony, but what about competition from money market funds?  In general, I have little confidence that the conclusions drawn from the model will be more generally valid. Second, unfortunately, that is all the paper delivers: some predictions about what would happen given a potential policy action. There is no empirical work to refute or confirm the theorizing. You admit yourself that the model may not be precise enough for all predictions to hold up, and the conclusions you think "may be" robust, seem a bit thin as the take away from a full-fledged academic article.


Observing a recent change in regulation to increase the portability of bank deposits and insurance policies, the authors investigate the impacts of this change in regulation on banks and insurers. The study is well motivated.

However, the work is not accomplished at a quantitative level that is compatible with this journal. Throughout the paper, the authors are satisfied with deterministic models, which do not seem to be sufficient for carrying out the proposed study. Some formulas are not formulated in a rigorous way. Taking (1) as an example, from the context around I do not see what T is. Moreover, why do those equalities (=0) and inequalities (>0) hold, or are they simply your assumptions for practical relevance? Taking (A.1) as another example, does it look too elementary to spend lines to derive such a derivative? Indeed, a freshman in any major of science is supposed to have learned the chain rule. These are just two examples, and such problems exist throughout the paper.

In view of the good motivation of the study, I would suggest that the authors resubmit the paper to another journal (like Journal of Risk and Insurance) which focuses more on insurance economics and less on quantitative contributions.
 




Responses to the referee
The points are taken up in the sequence of the report.

(1) I was not aware of major changes in the portability of financial contracts. To me, the examples seem in part esoteric and in part not fitting -- I don't know much about banking regulation and the market in India and Georgia, ...

Since so far few jurisdictions have introduced portability regulation, one cannot be choosy about examples to cite. 
 
(2)... the relevance of the referenced international agreements is difficult to assess in isolation, and the COI hikes in response to life settlements (liquidity of an asset) don't seem to match the story. 

The referee has a point w.r.t. the COI hikes in life settlements; more expaation is needed. The argument is that while consumers were targeted, the hikes were designed to make it more costly for intermediaries to purchase these settlements (typically for sale to another insurance company later on).  

(3) That said, entertaining the idea that changes in portability of financial contracts affects the institutions -- and potentially different institutions differently -- seems plausible.

Acknowledging differences in impacts contradicts point (15) below.

(4) However, and this is my most substantial concern, I believe the key research questions -- how changes in portability affect profitability of financial institutions, and whether there are fundamental differences between banking and insurance -- demand empirical answers. And especially so since the authors point to the empirical relevance of these changes.

Yes, of course. But where should that empirical evidence come from given that few jurisdictions have actually implemented this regulation? If taken at face value, this criticism implies that a purely theoretical paper should never be published.

(5) I do not think the theoretical investigation is convincing on its own, there are many loose ends. It may be sufficient for developing hypotheses to be tested in an empirical investigation. 

Yes, agreed, that is the objective. 

(6) Some of my concerns include:

- Micro-foundations are not modeled, the D's and V's and their partial effects are just presented, together with assumptions on exogeneity.

Would it really make sense to develop theory designed to predict that consumers react favorably to higher solvency (solvability, respectively)? Also, empirical evidence supporting this relationship is presented. Why should this not be sufficient?

 
(7) For instance, dD/dT (S - \bar{S}) seems to be driving much of the results. Is that warranted? 

This expression states that the response of depositors to an increase in portability is conditioned by whether or not the bank’s solvency exceeds a certain benchmark. The response is positive if it does but negative if the bank’s solvency is below the benchmark. All this does is to shift the benchmark for a positive response of depositors from zero (certainly an  extreme point) to e.g. the industry average (which is more realistic because this provides a comparison between banks)


(8) And isn't S a function of C? 

Good point – in principle; additional justification is needed here. A quick look at the comparative statics performed in the Appendix would have convinced the referee that obtaining reasonably clear predictions is already difficult (as so often), so something had to give. Moreover, part of capital is invested in assets that are not liquid enough to meet claims – that’s why solvency regulation distinguishes categories. One would therefore have to also categorize capital, further complicating the model.  
  
(9) Similarly, it seems that premiums should be function of \bar{V} -- think about premiums for a AAA vs. a B-rated insurer.

Well, where does an AAA rating vs a B- rating of a particular company come from? Certainly not derived from the industry average of solvability (as suggested by the referee) but from a company’s solvability compared to some benchmark – and that is exactly the specification of the model.


(10) Fundamentally, this seems to be about consumers' preferences for portability and solvency, so shouldn't that be the starting point?

Of course, but what is wrong with just using the empirical evidence rather than burdening the paper with this [see point (6) above]. 
(11) - There are various channels at work. Why solely focus on the solvency channel? What is the relationship between portability and solvency regulation? 

The empirical evidence cited suggests solvency is an important determinant of choice. And solvency regulation is unlikely to „bite“ as long as the bank is above the (industry) benchmark.  Banks below the benchmark satisfy the regulation by now, years after Solvency II. However, this still leaves room for despositors to prefer banks who (while complying with regulation) have higher solvency than average. And that‘s precisely what’s modelled.

(12) And the assumption that the IC's primary activity is underwriting is contentious (think e.g. about Warren Buffett's float).

OK, one can argue about ‚primary‘. However, the IC’s investment activity is taken into full account.


(13) - The authors ignore any convexities / second-order effects. 

Yes, but these ‚neglects‘ are explicitly stated as simplifications. They are also justified because without them, reasonably clear predictions could not be derived (once again, see the comparative statics in the Appendix).
 
(14) Could they matter, especially infra-marginally. Also, some laws of gravity seem problematic: If r_I > r_D and I can just expand coverage, is the problem bounded?

The problem is bounded. At some lengths (and with data on market concentration), the market for deposits is decribed as monopsonistic. This means that r_D incrreases when the bank seeks to expend its deposits, eating away at its margin. 

(15) More broadly, I have difficulty appreciating how profitability could "boost profitability." Don't regulations add constraints to the profit maximization problem, so how can profit increase relative to the unregulated case? 

This is a good point – in principle. More justification is indeed needed to elucidate what is different in the present context. As noted by the referee in point (3) above, regulation may affect firms differently. Some banks (insurers, respectively) may be poised to benefit because increased portability induces a migration of deposits (contracts, respectively) away from competitors. 
 
(16) Weren't firms free to offer more portable contracts from the start? Why does regulation changes this? 

Yes indeed – but why do banks (insurers, respectively) sanction customers who wish to change to begin with? One possible reason is that acquiring an extra customer is several times more costly than keeping one – but this is beyond the scope of the paper. And the banks of Georgia apparently did not resist portability regulation (presumably in the hope of building their fledgling business) – but the referee seems to dismiss this case [see point (1) above].  
