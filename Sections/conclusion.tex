\section{Conclusion and Outlook}
\label{sec:conclusion}

In view of an increasing tendency of regulatory authorities to increase the portability of bank deposits and insurance policies, this paper seeks to answer two research questions: What is the effect of increase portability on the profitability of banks and insurers? And, does this effect differ between banks and insurers? Two models are developed in which management maximizes profitability defined as return on equity capital through its choice of solvency (solvability, respectively), which is seen as the crucial decision variable. In both models, the initial level of solvency (solvability) turns out to be crucial because if it is above some benchmark set by the industry or the regulator, the bank (the insurer) attracts deposits (policies, respectively); if solvency is below the benchmark, deposits and policies migrate elsewhere.  Evidently, increased portability facilitates these transfers.


The difference in business models translates into differences in the way an increased portability affects the profitability of banks and insurers. Banks with initial solvency above the benchmark are found to see their profitability increase because by lowering their solvency, they shed both deposits that must be paid for and equity capital. Those with initial solvency below the benchmark stand to lose (somewhat if their capital base is large). Both of these effects become negligible for banks with few deposits (Conclusion 1). Insurance companies with an initial solvency above the benchmark are predicted to be hurt by an increased man-dated portability of their policies (to a substantial degree if their capital is small), contrary to banks. The reason is that the induced downward adjustment of their solvency causes a reduction in premium income which can be invested in the capital market exceeding the reduction in losses to be paid. This situation likely characterizes a majority of companies especially in Europe, who have adjusted their solvency to the level prescribed by \textit{Solvency II} regulation but continue to have a small capital base, at least compared to their premium volume. Also in contradistinction with banks, there is no subset of companies that are left unaffected (Conclusion 2).


Finally, the effect of a mandated increase in portability is amplified by several intervening variables, several of which concern insurers but not banks such as a long-tail risk portfolio or a high loss ratio. These distinctive properties serve to underline the fact that the same type of regulation may affect banks and insurers in quite different ways.

Of course, these findings are subject to a number of limitations. First, the two business mod-els may not be depicted accurately enough. In particular, the benchmark value below which an increase in portability causes a loss in deposits (policies, respectively) is left undetermined.  Related to this, existing solvency regulation may prevent a bank or an insurance company from lowering its solvency level further. Second, both predicted management responses and effects on profitability are conditional on several elasticities whose values have not been established or are unknown. This is true in particular of the elasticities with regard to the in-crease in portability affecting deposits in the case of banks and both premium income and losses in the case of insurance companies. Finally, for an overall evaluation of portability regulation, its impact on the profitability of banks and insurers would have to be pitted against the change in consumers' surplus.


Yet two insights of this contribution are likely to prove robust. The management of banks and insurance companies may react in different ways to a given regulation, depending on their initial solvency level in the present instance. In addition, the difference in business models be-tween banks and insurers is of considerable relevance for assessing the impact of uniform regulation on their respective profitability – and hence their likely stance with regard to that regulation.
