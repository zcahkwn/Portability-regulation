\section{Impact on banks of an increase in portability of policies}
\label{sec:impactbank}


% Consider a bank which holds a certain amount of its clients' deposits $D$.  Assume that $D$ is directly influenced by the bank's solvency status $S$ and portability $T$. 

% We model solvency $S$ to be a first-order decision variable, as it
% constitutes the crucial variable controlled by management. $S$ is governed by the bank's structuring of assets and liabilities with their risk characteristics, and can be equated to the likelihood of a shortfall (Leibowitz et al. \cite{Leibowitz1992AssetApproach}).

% Portability $T$ is exogenously determined and  $dT > 0$ denotes the mandated increase in portability. $dT$ can be broadly defined as added ease or smoothness in transferring a deposit to another bank.
% % (a policy to another insurance company, respectively).  
% This may mean a lower fee charged for closing an account, for wiring money from one bank to another, or for acquisition expense when switching the insurer (also, see Section \ref{sec:intro} again).
% % as expressed in Equation (\ref{eq:d}).

% % (see e.g. Saunders and Wilson \cite{Saunders1996ContagiousPeriod}, Park and Peristiani \cite{Park1998MarketDepositors}, Bennett, Hwa, and Kwast \cite{Bennett2015MarketCrisis}, as well as Iyer, Puri, and Ryan \cite{Iyer2016ARisk})


% % Therefore, $S$  
% % 
% % For concreteness, let $T$ be an index with values between 1 and 100; a marginal increase $dT$ thus amounts to one point on the scale.
% We express the amount of deposits held by a bank $D$ as a function of $S$ and $T$:

Consider a bank who pays for its deposits $D$ the rate $r_D$ and invests them at a rate of return $r_1 > r_D$. While $r_1$ is viewed as exogenous because banks compete with many investors on the global capital market, $r_D$ is an increasing function of deposits $D$, reflecting monopsony in their domestic markets. For instance, the four largest Australian banks held 78 percent of deposits in 2009, up from 65 percent in 2009 [OECD (2010), Table 4.1]; their overall share in the country's banking sector was also 78 percent between 2000 and 2007 [OECD (2010), Chart 4.1]. This overall share (which is likely to underestimate the share in deposits) ranged between a low 26 percent in the United States to 56 percent in the United Kingdom and on to 93 percent in Norway, 95 percent in Sweden, and 99 percent in Finland. In a majority of western countries, a bank therefore drives up the interest rate it has to pay on deposits when it expands. 

The bank's solvency status S constitutes the crucial variable controlled by management. It influences first of all the amount of deposits $D$ [see e.g. Saunders and Wilson (1996), Park and Peri-stiani (1998), Bennett, Hwa, and Kwast (2014), as well as Iyer, Puri, and Ryan (2016)]. Solvency can be equated to the likelihood of a shortfall (Leibowitz et al., 1992) or in terms of \ac{VaR} or \ac{EVaR} concepts. Whenever \ac{VaR} or \ac{EVaR}  increases, the solvency level can be said to decrease. Therefore, $S$ is governed by the bank's structuring of as-sets and liabilities with their risk characteristics. Let $dT > 0$ denote the mandated increase in portability, which can be broadly defined as added ease or smoothness in transferring a deposit to another bank (a policy to another insurance company, respectively).  This may mean a lower fee charged for closing an account, for wiring money from one bank to another, or for acquisition expense when switching the insurer (also, see \autoref{sec:intro} again). For concreteness, let $T$ be an index with values between 1 and 100; a marginal increase $dT$ thus amounts o one point on the scale. 
For a bank, one therefore has  $$D=D(S,T)$$ with

\begin{align}
\frac{\partial}{\partial S} D(S,T) \nonumber &>  0 \\
\frac{\partial ^2}{\partial S^2} D(S,T)&=0 \nonumber \\
\text{and} \quad \frac{\partial^2}{\partial S \partial T} D(S,T) &= 0 \quad \text{for simplicity; \nonumber} 
\end{align}
\begin{align}
    \frac{\partial D}{\partial T}(S-\overline{S})  &> 0 \label{eq:Sbar} 
\end{align}

% \\
% & \text{for simplicity we assume} & \frac{\partial^2 D}{\partial S^2}  = &  0 \\
% && \frac{\partial^2 D}{\partial S \partial T}  = & 0 

The last inequality implies that when solvency exceeds a certain benchmark $\overline{S}$ agreed upon by the industry or set by the regulatory authority), then $\partial D / \partial T >0$, indicating that the bank attracts deposits. Conversely, if its solvency falls short of $\overline{S}$, the inequality implies $\partial D / \partial T <0$, i.e. increased portability makes it easier for clients to turn to a safer bank (with higher solvency), which causes the bank to lose deposits. 

Profitability $R^B$ is defined as the bank's profit relative to its equity capital; it is assumed to be positive. Profit is derived from investing deposits $D$ at the rate $r_1$, minus interest payment for deposits $D$ at the rate $r_D$. Abstracting from operating costs and taxes, one has 
\begin{equation}
    R^B=\frac{\{r_I-r_d(D(S,T))\}D(S,T)}{C(S)} \label{eq:RB}
\end{equation}
with $\partial r_D / \partial D >0, \partial C / \partial S >0$, and all second-order derivatives equal to zero for simplicity. 



% % or in terms of value-at-risk (VaR) or expected value-at-risk (EVaR) concepts. A higher VaR or EVaR  implies a lower solvency level. 
% A higher $S$ suggests a lower likelihood of shortfall, which attracts more deposits $D$ (see e.g. Saunders and Wilson \cite{Saunders1996ContagiousPeriod}, Park and Peristiani \cite{Park1998MarketDepositors}, Bennett, Hwa, and Kwast \cite{Bennett2015MarketCrisis}, as well as Iyer, Puri, and Ryan \cite{Iyer2016ARisk}). Thus, intuitively, $D$ is positively related to $S$ as expressed in Inequality (\ref{eq:DS}).

The relationship $\partial C / \partial S >0$ takes into account that a higher solvency level also attracts equity capital; this is in accordance with the finding that banks' funding cost decreases with an increase in solvency (Aymanns et al., 2016). Management is assumed to maximize (expected) profit in a risk-neutral way, representing those highly diversified shareholders for whom idiosyncratic risks are of little concern.


The effect of an exogenous increase in portability (mandated by the regulator) on the profitability is given by

\footnotesize
\begin{equation}
    \frac{dR^B}{dT} = 
    \frac{
        \left\{ 
        -\frac{\partial r_D}{\partial D} 
        \left( \frac{\partial D}{\partial S} \frac{\partial S^*}{\partial T} + \frac{\partial D}{\partial T} \right) D
        + (r_I - r_D) 
        \left( \frac{\partial D}{\partial S} \frac{\partial S^*}{\partial T} + \frac{\partial D}{\partial T} \right) 
        \right\} C 
        - \left\{ r_I - r_D \left( D(S,T) \right) \right\} D \cdot \frac{\partial C}{\partial S} \frac{\partial S^*}{\partial T} \label{eq:dRB}
    }{C^2} 
\end{equation}
\normalsize

\noindent As shown in the Appendix [\autoref{eq:A_dRB}], this can be rewritten in terms of elasticities 

\small
\begin{align}
    \frac{dR^B}{dT} &= 
    -e(r_D, D) \cdot \left\{ 
        e(D, S) \cdot e(S^*, T) \frac{r_D D}{CT} 
        + e(D, T) \frac{r_D D}{CT} 
    \right\} \nonumber\\
    &+ \frac{r_I - r_D}{C} \bigg\{ 
        e(D, S) \cdot \frac{D}{S} \cdot e(S^*, T)\cdot \frac{S}{T} 
        + e(D, T) \cdot \frac{D}{T} 
    \bigg\} 
    - \frac{r_I - r_D}{C} \frac{D}{T} \cdot e(C, S) \cdot e(S^*, T) \label{eq:dRB_elasticities}
\end{align}
\normalsize

\noindent with $e(r_D, D) := \left( \frac{\partial r_D}{\partial D} / r_D \right) > 0$, $e(S^*, T) := \left( \frac{dS^*}{dT} (T / S) \right)$, $e(D, T) := \left( \partial D \partial T (T / D) \right)$, $e(C, S) := \left( \frac{\partial C / \partial S}{S / C} \right) > 0$, and $e(D, S) := \left( \frac{\partial D / \partial S}{S / D} \right) > 0$, respectively, which are all assumed to be constant. The term $e(S^*,T)$ symbolizes optimal adjustment by the bank, whose sign has to be determined by comparative-static analysis (see Appendix). Evidently, for banks with little equity capital, the absolute value of \autoref{eq:dRB_elasticities} becomes large.  The sign of $dR^B / dT$ importantly depends on the sign of $e(S^*,T)$, i.e. how banks adjust their solvency level to an increase in portability. 

According to \autoref{eq:A_dS*} of the Appendix,

\begin{equation}
    \frac{dS^*}{dT} =
    \begin{cases} 
        < 0 & \text{if } S > \bar{S}, \\
        > 0 & \text{if } S < \bar{S}, \\
        = 0 & \text{if } D \to 0. \label{eq:dS*}
    \end{cases}
\end{equation}

\noindent Moreover, one obtains (see also part \ref{tab:basic_effect} of \autoref{tab:predictions_compare} below):
\begin{equation}\label{eq:dRB_cases1}
    \frac{dR^B}{dT} =
    \begin{cases} 
        \gg 0 & \text{if } S > \bar{S} \text{ and } C \to 0, \quad \text{(a)}\\
        \ll 0 & \text{if } S < \bar{S} \text{ and } C \to 0, \quad \text{(b)}\\
        \to 0 & \text{if } D \to 0. \quad \text{(c)}
    \end{cases} 
\end{equation}



Case (a) obtains because a profit-maximizing bank lets deposits increase to a point where $e(r_D,D)$ becomes large\footnotemark, while holders of deposits are characterized by comparatively small values of $e(D,S)$ and $e(D,T)$ because they face transaction costs (barring a bank run). With $e(S^*,T)<0$, the first term of the bracket is positive. This is also true of the second term since shareholders, facing comparatively low transaction costs on the stock exchange, respond in a more marked way to a change in solvency than do depositors, thus $e(C,S)>e(D,S)$ and $e(C,S)>e(D,T)$. In case (b), the first term of the bracket is negative because $e(S^*,T)>0$ now; this is also true of the second term, again because of the comparatively high value of $e(C,S)$.

\footnotetext{For the sake of illustration, let the bank determine $D$ directly and treat equity as exogenous. Then the derivative of the numerator in \autoref{eq:RB} w.r.t. $D$ solves for $r_1-\partial r / \partial D \cdot D- r_D = r_1 - r_D \{1+e(r_D,D)\}=0$. With $r_1=0.1$ and $r_D=0.02$, say, this implies $e(r_D,D)=4$.}

Therefore, banks with a small equity base stand to benefit greatly if their initial solvency level is above the benchmark by shedding deposits [recall the high value of  ], while they suffer greatly if their initial solvency level is below the benchmark. The first situation [case (a) of \autoref{eq:dRB_cases1}] may well be characteristic for many European banks who achieved the solvency level required by Basel II regulation while their equity capital is still small (at least relative to their deposits) (Humblot \cite{Humblot2018TheImprove}). Those with few deposits are hardly affected, a situation typical of start-ups in the banking industry.

As to highly capitalized banks, one has, using \autoref{eq:dS*} again,

\begin{equation}
    \frac{d R^B}{d T}=\left\{\begin{array}{l}
    \geq 0 \text { if } S>\bar{S} \text { and } C \rightarrow \infty , \quad \text{(d)} \\
    \leq 0 \text { if } S<\bar{S} \text { and } C \rightarrow \infty , \quad \text{(e)} \\
    \rightarrow 0 \text { if } D \rightarrow 0 , \quad \text{(f)} .
    \end{array}\right. \label{eq:dRB_cases2}
\end{equation}


In cases (d) and (e), the signs can be determined in the same way as in cases (a) and (b) of \autoref{eq:dRB_cases1} above.  Therefore, among highly capitalized banks, again only those with an initial solvency level below the benchmark may suffer somewhat from an increase in portability; those with a high initial solvency level even stand to benefit, while those with few deposits again are hardly affected.


\textbf{Conclusion 1:} A mandated increase in portability of deposits is predicted to affect the profitability of banks depending on their initial level of solvency. Those with initial solvency above the benchmark have a chance to benefit by lowering it and hence the amount of their deposits, thus reducing their cost of funding [substantially so in the case of case (a) above]. Those with solvency below the benchmark but a large capital base stand to see their profitability fall somewhat [case (e)], while those with few deposits are hardly effected [cases (c) and (f)].

\begin{table}[h!]
\centering
\caption{Comparison of predictions} \label{tab:predictions_compare}
\scriptsize
% \renewcommand{\arraystretch}{2}
    \begin{subtable}{\textwidth}
        \captionsetup{justification=raggedright, singlelinecheck=false,font=small}
        \caption{Basic effects} \label{tab:basic_effect}
        \renewcommand{\arraystretch}{2}
        \scriptsize
        \begin{tabular}{|p{7cm}|p{7cm}|}
            \hline 
            \textbf{Banks} & \textbf{Insurers} \\ \hline
            $\frac{dS^*}{dT}=\begin{cases}
                < 0 & \text{if } S > \bar{S}, \\ 
                > 0 & \text{if } S < \bar{S}.
            \end{cases}$ 
            & $\frac{dV^*}{dT}= \begin{cases}
                < 0 & \text{if } V > \bar{V}, \\ 
                > 0 & \text{if } V < \bar{V}.
            \end{cases}$ \\ \hline 
            $\frac{dR^B}{dT}=\begin{cases} 
                \gg 0 & \text{if } C \to 0 \text{ and } S > \bar{S} \quad \text{(a)}, \\ 
                \ll 0 &\text{if } C \to 0 \text{ and } S < \bar{S} \quad \text{(b)}\label{eq:b}, \\ 
                \to 0 & \text{if } D \to 0 \hspace{1.75cm}
                \text{(c)}.
                \end{cases}$ 
            & $\frac{dR^{IC}}{dT}=\begin{cases} 
                \ll 0 & \text{if } C \to 0 \text{ and } V > \bar{V} \quad \text{(a')}, \\ 
                \gg 0 & \text{if } C \to 0 \text{ and } V < \bar{V} \quad \text{(b')}.
            \end{cases}$ \\ \hline
            $\frac{dR^B}{dT}=\begin{cases} 
                \geq 0 & \text{if } C \to \infty \text{ and } S > \bar{S} \quad \text{(d)}, \\ 
                \leq 0 & \text{if } C \to \infty \text{ and } S < \bar{S} \quad \text{(e)}, \\ 
                \to 0 & \text{if } D \to 0 \hspace{1.9cm} \text{(f)}.
            \end{cases}$ 
            & $\frac{dR^{IC}}{dT}=\begin{cases} 
                \leq 0 & \text{if } C \to \infty \text{ and } V > \bar{V} \quad \text{(d')}, \\ 
                \geq 0 & \text{if } C \to \infty \text{ and } V < \bar{V} \quad \text{(e')}.
            \end{cases}$ \\ \hline 
        \end{tabular}
    \end{subtable}
    
    \vspace{5mm}
    \begin{subtable}{\textwidth}
        \captionsetup{justification=raggedright, singlelinecheck=false, font=small}
        \caption{Amplifying effects: The effect of portability on profitability is the more marked, …} \label{tab:amplifying_effects}
        \renewcommand{\arraystretch}{2}
        \scriptsize
        \begin{tabular}{|p{7cm}|p{7cm}|}
            \hline 
            \textbf{Banks} & \textbf{Insurers} \\ \hline 
            The smaller the bank's equity capital $C$ & -- \\ \hline 
            The lower its initial degree of portability $T$ & The lower the initial level of portability $T$ (which is very low for ICs to begin with) \\ \hline 
            The higher the rate of interest $r_D$ paid on deposits & -- \\ \hline 
            The higher its interest margin $(r_I-r_D)$& --\\ \hline 
            The stronger the reaction of depositors to a change in solvency $e(D,S)$ & The stronger the reaction of policyholders to a change in solvability $e(P,V)$ \\ \hline 
            -- & The more strongly losses react to the increase in portability $e(L,T)$ \\ \hline 
            The higher $|e(S^*,T)|$, i.e. the more marked the bank's solvency response to increased portability & The higher $|e(V^*,T)|$, i.e. the more marked the insurer's solvability response to increased portability \\ \hline 
            The stronger the reaction of investor to a change in solvency $e(C,S)$ & The stronger the reaction of investors to a change in solvability $e(C,V)$ \\ \hline 
            The stronger the reaction of depositors to the increase in portability $e(D,T)$, provided $e(S^*,T)>0$ & The stronger the reaction of policyholders to the increase in portability $e(P,T)$ \\ \hline 
            -- & The higher the funds generating factor $k$, i.e. more the risk portfolio is of the long-tail type, compounded by a high rate of return on investment $r_I$ \\ \hline 
            -- & The higher the loss ratio $L/P$ \\ \hline
        \end{tabular}
    \end{subtable}
\end{table}

Note that \textbf{Conclusion 1} predicts a fall in profitability among some banks although their management adjusts their solvency in an optimal way by assumption. A suboptimal adjustment would cause them to suffer an even greater loss in their profitability. As a piece of very preliminary evidence, Terabank of eastern European Georgia likely has favorable solvency status following recent banking regulation combined with a mall capital base since it is a start-up. This bank indeed seems to benefit considerably from the national regulator's move to facilitate portability (Terabank,2018).
% Table 1. Comparison of predictions
% Banks	Insurers
% A. Basic effects

% B. Amplifying effects: The effect of portability on profitability is the more marked, …
% the smaller the bank's equity capital
% --
% the lower its initial degree of portability T	the lower the initial level of portability   (which is very low for ICs to begin with)
% the higher the rate of interest   paid on deposits	--
% the higher its interest margin
% --
% the stronger the reaction of depositors to a change in solvency
% the stronger the reaction of policyholders to a change in solvability
% --	the more strongly losses react to the increase in portability
% the higher   , i.e. the more marked the bank's solvency response to increased portability	the higher  , i.e. the more marked the insurer's solvability response to increased portability
% the stronger the reaction of investors to a change in solvency
% the stronger the reaction of  investors to a change in solvability
% the stronger the reaction of depositors to the increase in portability , provided
% the stronger the reaction of policyholders to the increase in portability
% --	the higher the funds generating factor  , i.e. more the risk portfolio is of the long-tail type, compounded by a high rate of return on investment
% --	the higher the loss ratio

More generally, the impact of portability on the bank's profitability depends on several intervening variables. Focusing on the numerator of \autoref{eq:A_dRB_pde}\footnote{The impacts on the denominator causing a movement away from zero serve to reduce the absolute value of the changes discussed, while a movement towards zero must be minor lest the denominator change sign, transforming a maximum into a minimum.}, one finds $| dR^B / dT|$  to be positively related to several factors that are listed in part B of \autoref{tab:predictions_compare}, among them not only the interest margin $(r_1-r_D)$  but also the rate of interest $r_D$  paid on deposits.
