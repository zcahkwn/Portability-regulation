\section{Impact on insurers of an increase in portability of policies}
\label{sec:impactins}

In contradistinction to a bank, which mainly acts as an investor of its deposits, an \acl{IC}'s (\ac{IC}'s) primary activity is risk underwriting; capital investment becomes possible when losses paid fall short of premium income. Accordingly, an \ac{IC}'s rate of return $R^{IC}$ has two components. Profits from investment activity derive first from capital $C$ being invested at a rate of return $r_G$, reflecting the fact that insurers almost everywhere are legally obliged to hold government bonds (for simplicity, capital is equated to reserves here).  Second, current premium income from risk underwriting makes funds available for investment according to the so-called funds-generating factor $k$ (Cummins and Phillips \cite{cummins2000}). The higher $k$, the longer the lag between premiums received and claims paid. These funds can be invested at the same rate of return $r_I$  as applied to banks.


The insurer also derives profit from risk underwriting itself. This component is simply given by the difference between current premium income $P$ (which increases with solvency $V$ (Cummins and Sommer \cite{cummins1996}) and losses paid $L$ (which are viewed as exogenous except for portability of policies $T$). Abstracting from operating costs and taxes again, $R^{IC}$ can be expressed as follows,

\begin{align}
    R^{IC} &= 
    \frac{\{r_G C(V) + k \cdot r_I P(V, T)\} + \{P(V, T) - L(T)\}}{C(V)} \nonumber\\
    &= r_G + \frac{(1 + k_I) \big(P(V, T) - L(T)\big)}{C(V)} \label{eq:RIC}
\end{align}

\noindent with $\partial C / \partial V>0$, $\partial P / \partial V>0$, $\partial P / \partial T(V-\bar{V}>0$, $\partial L / \partial T(V-\bar{V}>0$, and all second-order derivatives equal to zero for simplicity.

Portability increases premium volume (but also losses) if the insurer's solvency $V$ is above some benchmark $\bar{V}$ but depresses premium volume (and with it losses) if $V$ is below $\bar{V}$.

In view of \autoref{eq:RIC}, the effect of an increase in portability on profitability is given by
\begin{equation}
\frac{dR^{IC}}{dT} = 
\frac{
    \bigg\{
        (1 + k_I) 
        \left( 
            \frac{\partial P}{\partial V} \frac{dV^*}{dT} 
            + \frac{\partial P}{\partial T} 
        \right)
            - \frac{\partial L}{\partial T}
    \bigg\}C
        - \{ (1 + k_I) P - L \}
        \frac{\partial C}{\partial V} \frac{dV^*}{dT}
}{C^2}. \label{eq:dRIC}
\end{equation}

According to \autoref{eq:A.9} of the Appendix, this can be rewritten as

\begin{align}
    \frac{dR^{IC}}{dT} &= 
    \frac{(1 + k_I) \, P}{CT} 
    \bigg\{ 
        e(P, V) \cdot e(V^*, T) + e(P, T) - \frac{1}{1 + k_I} e(L, T) \frac{L}{P} 
    \bigg\} \nonumber\\
    &- \frac{1}{CT} 
    \bigg\{ 
        (1 + k_I) P - L 
    \bigg\} e(C, V) \cdot e(V^*, T), \label{eq:dRIC_elasticity}
\end{align}

\noindent using elasticity notation with $e(C,V):= (\partial C/ \partial V)>0$, $e(P,V):= (\partial P / \partial V)(V/P)>0$, $e(L,T):= (\partial L / \partial T) (T/L)>0$, $e(P,T):=(\partial P/ \partial T)(T/P)$, assuming constancy again, and with signs determined according to \autoref{eq:RIC}. The sign of    is determined in the Appendix.

In fact, according to \autoref{eq:A_dV*} of the Appendix, one has

\begin{equation}
    \frac{dV^*}{dT} =
    \begin{cases} 
        < 0 & \text{if } V > \bar{V} \\ 
        > 0 & \text{if } V < \bar{V}
    \end{cases}. \label{eq:dV*}
\end{equation}

Using these results in \autoref{eq:dRIC}, one obtains for \acp{IC} with a low amount of capital ($C \rightarrow 0$, see also part A of \autoref{tab:predictions_compare})\footnote{In the case of $V<\bar{V}$, the assumption is that the adjustment is not so strong as to move solvency above the benchmark.}.

\begin{equation}
    \frac{dR^{IC}}{dT} =
    \begin{cases} 
        \ll 0 & \text{if } V > \bar{V} \text{ and } C \to 0 \quad \text{(a')} \\ 
        \geq 0 & \text{if } V < \bar{V} \text{ and } C \to 0 \quad \text{(b')}
    \end{cases}. \label{eq:dRIC_cases1}
\end{equation}


Case (a') derives from the fact that the first term is negative (with $V>\bar{V}$, $e(V^*T,\bar{V}<0$, $e(P,T)>0 $ and $e(L,T)>0$); the second term, while positive since $(1+kr_I)P-L>0$ under normal circumstances is comparatively small since $P \simeq L$. Case (b') follows from the fact that the first term in \autoref{eq:dRIC_elasticity} turns positive as long as $e(V^*,t)>0$ is of a comparable magnitude as $e(P,V)$ and $e(P,T)$; the second term is positive.

For highly capitalized insurers ($C \rightarrow \infty$), one has
\begin{equation}
    \frac{dR^{IC}}{dT} =
    \begin{cases} 
        \leq 0 & \text{if } V > \bar{V} \text{ and } C \to \infty \quad \text{(d')} \\ 
        \geq 0 & \text{if } V < \bar{V} \text{ and } C \to \infty \quad \text{(e')}
    \end{cases}.\label{eq:dRIC_cases2}
\end{equation}

Arguably, case (a') applies to a majority of \acp{IC}, at least in Europe. While having adjusted their solvency to the higher level required by \textit{Solvency II} regulation, many still are not disposing of a large amount of capital, at least relative to their premium income (Umapathy \cite{Umapathy2016MatchingFit}).


\textbf{Conclusion 2:} A mandated increase in portability of insurance portfolios is predicted to affect the profitability of insurance companies differently, too. Those with initial solvency above the benchmark stand to suffer [greatly if their capital base is small, case (a') above]; they are likely to constitute the majority.  Conversely, it is precisely those with a solvency level below the regulatory benchmark will see their profitability increase [greatly if their capital base is small, case (b') and somewhat if it is large case (e')] thanks to an upward adjustment of their solvency.

Again, Conclusion 2 predicts a fall in profitability among some insurance companies although their management adjusts their solvency in an optimal way; a suboptimal adjustment would cause them to suffer an even greater loss in their profitability. Note that contrary to Conclusion 1 for the banks, Conclusion 2 does not refer to a category of insurance companies that is unaffected by the mandated increase in portability of their policies.

Again, a piece of preliminary evidence comes from the resistance of Indian health and life insurance companies against the IRB's intention to increase portability of policies. While the industry leaders HDFC, ICICI, and SBILife are likely to have a solvency level above the benchmark, they are also likely to be low in equity capital, having experienced premium growth of between 14 and 25 percent during the past few years (India Brand Equity Foundation \cite{IndiaBrandEquityFoundation2020InsuranceIndia}). This is likely to put them in category (a') of \autoref{eq:dV*}.


More generally, the impact of portability on the IC's profitability depends on several intervening variables. Focusing on the numerator of \autoref{eq:dRIC_elasticity}\footnote{Again, the impacts on the denominator causing a movement away from zero serve to reduce the absolute value of the changes discussed, while a movement towards zero must be minor lest the denominator change sign, transforming a maximum into a minimum.}, one finds $|dR^{IC}/dT|$ to be positively related to a low initial degree of portability (a traditional characteristic of insurance), and a long-tail risk portfolio (see part B of \autoref{tab:predictions_compare}), and a high loss ratio, properties that are irrelevant for banks.
