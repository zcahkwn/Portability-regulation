\subsection{Banks}
\label{sec:banks}
Elastities and sign and magnitude of $dS^*/dT$

For comparing the magnitudes of effects, it is helpful to use elasticity notation for \autoref{eq:dRB} of the text, which is repeated here,

\scriptsize
\begin{align}
\frac{dR^B}{dT} &= 
\frac{\left\{
    \frac{\partial r_D}{\partial D}
    \left( 
        \frac{\partial D}{\partial S} \frac{\partial S^*}{\partial T} 
        + \frac{\partial D}{\partial T} 
    \right)D
    + (r_I - r_D)
    \left(
        \frac{\partial D}{\partial S} \frac{\partial S^*}{\partial T} 
        + \frac{\partial D}{\partial T}
    \right)
\right\} C - \left\{ r_I - r_D(D(S, T)) \right\} D \cdot \frac{\partial C}{\partial S} \frac{\partial S^*}{\partial T}}{C^2} \nonumber \\
&= - \frac{\partial r_D}{\partial D}\frac{D}{r_D}\cdot \frac{r_D}{D}
    \left( 
        \frac{\partial D}{\partial S} \frac{S}{D}\cdot \frac{D}{S} \cdot \frac{\partial S^*}{\partial T}\frac{S}{T}\cdot \frac{D}{C}
        + \frac{\partial D}{\partial T} \frac{T}{D}\cdot \frac{D}{C}
    \right) 
    + \frac{r_I - r_D}{C} 
    \left( 
        \frac{\partial D}{\partial S}\frac{S}{D} \cdot \frac{D}{S}\cdot \frac{d S^*}{dT}\frac{T}{S}\cdot \frac{S}{T} 
        + \frac{\partial D}{\partial T}\frac{T}{D}\cdot \frac{D}{T}
    \right) \nonumber \\
&\quad - \frac{r_I - r_D}{S} \cdot \frac{D}{S} \cdot \frac{\partial C}{\partial S} \frac{S}{C} \cdot \frac{\partial S^*}{\partial T}\frac{T}{S} \cdot \frac{S}{T}\nonumber \\
&= - e(r_D, D) \cdot \left\{ 
        e(D, S) \cdot e(S^*, T) \frac{r_D D}{CT} 
        + e(D, T) \frac{r_D D}{CT} 
    \right\} + \frac{r_I - r_D}{C}
    \left\{
        e(D, S) \frac{D}{S} \cdot e(S^*, T) \frac{S}{T} 
        + e(D, T) \frac{D}{T}
    \right\} \nonumber \\
&\quad - \frac{r_I - r_D}{C} \frac{D}{T} \cdot e(C, S) \cdot e(S^*, T). \label{eq:A_dRB}
\end{align}
\normalsize

\noindent after introducing the elasticities $e(r_D,D):=(\partial r_D / \partial D)(D/r_D)>0$, $e(S^*,T):=(dS^*/dT)$, $e(C,S):=(\partial C/\partial S)(S/C)>0$ and $e(D,S):=(\partial D/\partial S)(S/D)>0$. 

Next, for deriving the bank's optimal adjustment to an increase in portability $dS^*/dT$, one needs to start from the first-order condition (FOC) w.r.t. the level of solvency. From \autoref{eq:RB}, one has, dropping arguments where possible,

\begin{align}
\frac{dR^B}{dS} &= 
\frac{\left\{
    -\frac{\partial r_D}{\partial D} \frac{\partial D}{\partial S} D 
    + (r_I - r_D) \frac{\partial D(S, T)}{\partial S}
\right\} C - (r_I - r_D) D \frac{\partial C(S)}{\partial S}}{C^2} \nonumber\\
&= - \frac{\partial r_D}{\partial D} \frac{\partial D}{\partial S} D 
+ (r_I - r_D) \left( 
    \frac{\partial D}{\partial S} C - \frac{\partial C}{\partial S} D 
\right) = 0 \label{eq:A_dRB_FOC}
\end{align}

\noindent after multiplication by $C^2>0$, noting that \autoref{eq:A_dRB_FOC} is sufficient for describing both the optimum and the transition from one optimum to another (see below). An interior solution to \autoref{eq:A_dRB_FOC} is ascertained since both $\partial D/\partial S>0$ and $\partial C/ \partial S>0$, while $(r_I-r_D)>0$.

The comparative-static equation describing optimal adjustment to an exogenous change $dT>0$ reads, 

\begin{equation}
    \frac{\partial^2 R^B}{\partial S^2}dS^* + \frac{\partial^2 R^B}{\partial S \partial T}dT=0 \label{eq:A_dRB_pde}
\end{equation}

\noindent which can be solved for 

\begin{align}
\frac{dS^*}{dT} &= -\frac{\frac{\partial^2 R^B}{\partial S \partial T}}{\frac{\partial^2 R^B}{\partial S^2}} \nonumber\\
&= \frac{
    \left(
        \frac{\partial^2 r_D}{\partial D \partial S}\frac{\partial D}{\partial S} + \frac{\partial r_D}{\partial D} \frac{\partial^2 D}{\partial S \partial T}
    \right) D 
    + \frac{\partial r_D}{\partial D} \frac{\partial D}{\partial S} 
    \left( \frac{\partial D}{\partial T} + \frac{\partial D}{\partial S} \frac{\partial S^*}{\partial T} \right) D 
    + \frac{\partial r_D}{\partial D} \frac{\partial D}{\partial S} D
}{\frac{\partial^2 R^B}{\partial S^2}} \nonumber \\
&\quad - \frac{
    \left\{ (r_I - r_D) \left( \frac{\partial^2 D}{\partial S \partial T} C - \frac{\partial C}{\partial S} \frac{\partial D}{\partial T} D \right) \right\}
}{\frac{\partial^2 R^B}{\partial S^2}} \nonumber \\
&= \frac{
    \frac{\partial r_D}{\partial D} \frac{\partial D}{\partial S} 
    \left( \frac{\partial D}{\partial T} + \frac{\partial D}{\partial S^*} \frac{\partial S^*}{\partial T} \right) 
    D + \frac{\partial r_D}{\partial D} \frac{\partial D}{\partial S} D 
    + \frac{\partial r_D}{\partial D} \frac{\partial D}{\partial S} \frac{\partial D}{\partial T} 
    + (r_I - r_D) \frac{\partial C}{\partial S} \frac{\partial D}{\partial T} D
}{\Delta} \label{eq:A_dS*_expantion}
\end{align}

\noindent due to the assumption in the main text that all second-order derivatives are equal to zero and with $\delta := \partial^2 R^B / \partial S^2 <0$ for a maximum. \autoref{eq:A_dS*_expantion} can be rewritten to become

\begin{equation}
\frac{dS^*}{dT} \bigg\{ 1 - \frac{\frac{\partial r_D}{\partial D} \left( \frac{\partial D}{\partial S} \right)^2 D }{\Delta}\bigg\}
= \frac{\frac{\partial r_D}{\partial D} \frac{\partial D}{\partial S} \frac{\partial D}{\partial T} D
+ \frac{\partial r_D}{\partial D} \frac{\partial D}{\partial S} D
+ \frac{\partial r_D}{\partial D} \frac{\partial D}{\partial S} \frac{\partial D}{\partial T} D+ (r_I - r_D) \frac{\partial C}{\partial S} \frac{\partial D}{\partial T} D}{\Delta} 
\end{equation}

\noindent which can be solved for $dS^*/dT$,

\begin{equation}
\frac{dS^*}{dT} = 
\frac{\dfrac{\partial D}{\partial T} \bigg\{ \left( \dfrac{\partial r_D}{\partial D} \dfrac{\partial D}{\partial S} + (r_I - r_D) \dfrac{\partial C}{\partial S} \right) D \bigg\} 
+ \dfrac{\partial r_D}{\partial D} \dfrac{\partial D}{\partial S} D}{\Delta - \dfrac{\partial r_D}{\partial D} \left( \dfrac{\partial D}{\partial S} \right)^2 D}. 
\end{equation}

Thus, noting that the denominator is negative and recalling that $\partial D/ \partial T>0 $ is $S>\bar{S}$ but $\partial D / \partial T <0$ if $S< \bar{S}$, one obtains

\begin{align}
    \frac{dS^*}{dT}=
    \begin{cases}
        &<0 \text{ if } S>\bar{S} \\
        &>0 \text{ if } S<\bar{S} \\
        &=0 \text{ if } D \to 0.
    \end{cases} \label{eq:A_dS*}
\end{align}

\noindent This is \autoref{eq:dS*} of the text.