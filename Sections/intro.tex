\section{Introduction}
\label{sec:intro}

With occupational and geographical mobility on the increase both domestically and internationally, portability of financial assets has become ever more important to consumers. In the case of banks, clients may want to move their deposits or their mortgages; in the case of insurers, clients usually wish to retain their health insurance coverage and pension benefits when switching employers. However, such changes are often fraught with considerable transaction costs: With their old bank and insurer, clients may face penalties for early withdrawal of a deposit or partial loss of benefits; with their new financial institutions, costs for credit approval and medical certificates can incur. Regulatory authorities tend to see these frictions as welfare-reducing, causing them to intervene.


In the case of banks, regulatory intervention is a relatively recent phenomenon. For example, it was only a few years ago that the \ac{RBI} urged banks to permit the transfer of deposits from one of their branches to another free of charge \cite{PTI2012RBIPortability}. However, the banking industry apparently resisted this move, causing \ac{RBI} \cite{ReserveBankofIndia2015MasterBanks} to issue a Master Circular on Customer Service in Banks mandating banks to disclose its penal rates for early bulk withdrawal of deposits amounting to less than 1 crore (= USD 141,000 as of 2019 exchange rates); however, it still permitted them to set their own penal rates. Two years later, \ac{RBI} cautioned banks against charging higher rates on services in the context of account portability (Sharma \cite{Sharma2017RBITimes}). Recently, the National Bank of Georgia, in an attempt to grow the country's banking industry, introduced the portability of deposits. Indeed, Georgia's Terabank offers help to organize the transfer of a deposit to itself (The Financial \cite{TheFinancial2018TerabankTerabank}). Similarly, South Africa plans to adopt portability of bank deposits in anticipation of an expansion of internet banking (BusinessTech \cite{BusinessTech2018SouthReport}).


Historically, concerns with the portability of insurance benefits predated those with portability of bank deposits. As early as 1996, the US Health Insurance and Portability and Accountability Act was promulgated in the aim of improving continuity of health insurance coverage, which had typically been lacking when a worker changed employers. The \ac{IRDAI} recently mandated health insurers to facilitate portability; in particular, the waiting period for coverage to be granted after a switch cannot exceed the residual under the previous policy (\ac{IRDAI} \cite{IRDAI2020GuidelinesPolicies}), defying skepticism voiced by the industry (Bhaskaran \cite{Bhaskaran2017WillWork}).


While these regulations are within-country, an example of cross-border regulation affecting both banks and insurers is the portability of pensions (so-called superannuation benefits) between Australia and New Zealand (Treasury of Australia \cite{TreasuryofAustraliaSuperannuationZealand}). The European Commission \cite{EuropeanCommission2017ProposalProduct} has also planned regulation ensuring international portability of pension benefits in the guise of the \ac{PEPP} proposal.


In sum, public regulation designed to facilitate portability has typically been resisted by both banks and insurers, but not all of them.  This observation gives rise to two research questions. First, what are the characteristics of banks and insurers respectively who are (dis)advantaged by this type of regulation? Second, do insurers differ systematically from banks in this respect? Both questions call for an analysis of the impacts of portability regulation on the profitability of banks and insurers. The second question is motivated by an earlier finding that solvency regulation affects banks and insurers differently \cite{Zweifel2015}. In this paper, we attempt to answer those questions with a simple, static model.


The plan of the remainder of this paper is as follows. In \autoref{sec:impactbank}, the management of a bank seeking to maximize its profitability (defined as return on equity capital) is modelled. The aim is to determine the impact of a mandated increase in the portability of its deposits, which depends on the management's response to the regulation.  The bank's solvency level constitutes the crucial decision variable because it influences both the amount of deposits and equity capital. If it falls short of the industry-wide benchmark, management is predicted to adjust it upward, which attracts additional deposits as well as equity capital. However, the impact on profitability is negative, especially so if the initial capital base is small [see left-hand side of \autoref{tab:predictions_compare}, case (b)]. The impact of portability regulation on the profitability of an insurer is addressed in \autoref{sec:impactins}. Again, management is predicted to adjust the company's solvability level upward in response to an increase in portability if solvability is initially below the industry benchmark. However, here the impact on profitability is positive, especially so if the initial capital base is small [see right-hand side of \autoref{tab:predictions_compare}, case (b')].


Moreover, a number of factors intervene to amplify the impact on profitability factors such as a strong reaction of investors to a change in solvency (solvability, respectively). Some of them differ between banks and insurers; for the latter, it is important whether or not their risk portfolio is of the long-tail type (see part \ref{tab:amplifying_effects} of \autoref{tab:predictions_compare}). These differences result from the difference in business models; therefore, a given regulation may well affect banks and insurers differently because of the difference in their business models. \autoref{sec:conclusion} offers a summary as well as a few final remarks and conclusions.
