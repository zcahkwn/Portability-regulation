\begin{abstract}
During the past few years, regulation designed to increase the portability of bank deposits and insurance policies has been promulgated, giving rise to two research questions: What is the impact of this regulation on the profitability of banks and insurers, respectively? And do these impacts differ in view of the difference in business models?  A management who seeks to maximize profitability is predicted to lower its solvency level if is initially above some benchmark and to increase it otherwise. In the case of a bank, one with an initial solvency level below the benchmark is predicted to face profitability decline after mandated portability increase especially if equity capital is low; in the case of an insurer, profitability is boosted. This difference is due to the fact that premium income increases more than do losses in response to higher solvability. This creates scope for extra returns from investment, an effect that is absent from banks. \\

% During the past few years, regulation designed to increase the portability of bank deposits and insurance policies has been promulgated, giving rise to two research questions: What is the impact of this regulation on the profitability of banks and insurers, respectively? And do these impacts differ in view of the difference in business models?
% To this end, we model the impact of portability on the profitability of banks and insurers conditioned on the entity initial solvency level. A management who seeks to maximize profitability is predicted to lower its solvency level if is initially above some benchmark and to increase it otherwise. 
% In the case of a bank, one with an initial solvency level below the benchmark is predicted to face profitability decline after mandated portability increase especially if equity capital is low; in the case of an insurer, profitability is boosted. This difference is due to the fact that premium income increases more than do losses in response to higher solvability. This creates scope for extra returns from investment, an effect that is absent from banks. 
% Thus, we demonstrate that the difference in business models between banks and insurers is of considerable relevance for assessing the impact of uniform regulation on their respective profitability---and hence their stance with regard to that regulation.

% We investigate why some banks (insurers) resist an obligatory portability increase in deposits (insurance policies) whereas others embrace it. 
% Our model suggests that mandated portability increase benefits banks (insurers) with an above-benchmark (below-benchmark) initial solvency level benefit, while hurting the rest. 


{\bf JEL codes:} G15, G21, G28, L51

{\bf Keywords:} Portability Regulation, Solvency, Solvability, Banks, Insurers 
% Trading, Investor behavior, Learning, Rationality
\end{abstract}